\documentclass{beamer}
\usepackage{amssymb}
\usepackage{amsmath}
\usepackage{stmaryrd}
\usepackage{graphicx}
\usepackage{isabelle}
\usepackage{isabellesym}

\usepackage{algorithmic}
\usepackage[linesnumbered,ruled,vlined]{algorithm2e}

\usepackage{ctex}

%%%%%%%%%%%% For Isabelle code
\newlength{\fminilength}
\newsavebox{\fminibox}
\newenvironment{fmini}[1][\linewidth]
  {\setlength{\fminilength}{#1\fboxsep-2\fboxrule}%
   \vspace{2ex}\noindent\begin{lrbox}{\fminibox}\begin{minipage}{\fminilength}%
   \mbox{ }\hfill\vspace{-2.5ex}}%
  {\end{minipage}\end{lrbox}\vspace{1ex}\hspace{0ex}%
   \framebox{\usebox{\fminibox}}}

\newenvironment{specification}
{\noindent\scriptsize
\tt\begin{fmini}\begin{tabbing}X\=X12345\=XXXX\=XXXX\=XXXX\=XXXX\=XXXX
\=\+\kill} {\end{tabbing}\normalfont\end{fmini}}

\usetheme{Warsaw}
\def \twoSpaces {\ \ }

\def \oneSpace {\ }

\def \oneSpace {\ }
\def \eqc {= }
\def \andc {\wedge }
\def \negc {\neg }
\def \orc {\vee }

\begin{document}

%%-------------------------------------------------


\title{ A Novel Approach to Parameterized verification of Cache Coherence Protocols}
%\titlerunning{A Novel Approach to Parameterized verification of Cache Coherence Protocols}

\author{Yongjian Li\inst{1} \and  Kaiqiang~ Duan\inst{1} \and Yi~Lv\inst{1} \and Jun Pang\inst{2} \and Yi Lv\inst{1}
\and Shaowei~Cai\inst{1}}

%\authorrunning{Li et al.}

\institute{
State Key Laboratory of Computer Science, China \and
Computer Science and Communications, University of Luxembourg, Luxembourg
}
\frame{\titlepage}
%%-------------------------------------------------

\begin{frame}\frametitle{Problem of Parameterized Verification}
Consdier a protocol $P$, a property $Inv$
\begin{itemize}
\item  $P(N) \models Inv$ for any $N$
\item not just for a single protocol instance  $P(c)\models Inv$
\item Our opinion: parameterized verification is a theorem proving problem
\end{itemize}
\end{frame}

\begin{frame}\frametitle{State of Arts}
\begin{itemize}
\item CMP:  parameter abstraction and parameter abstraction
\item Proposed, by McMillan, elaborated by Chou, Mannava, and Park (CMP) , and formalized
by Krstic
\item construction of an abstract instance which can simulate any protocol instance
\item human provides auxiliary invariants (non-interference lemmas)
\end{itemize}
\end{frame}


\begin{frame}\frametitle{State of Arts }
\begin{itemize}
\item  invisible invariants,
\item  auxiliary invariants are computed from reachable state set in a finite protocol
instance $P(c)$
\item raw formula translated from BDD
\item  the reachable state set can't be enumerated, e.g., the FLASH protocol
\end{itemize}
\end{frame}



\begin{frame}\frametitle{Two central and difficult problems}
\begin{itemize}
\item  searching auxiliary invariants is not automatic
\item  soundness problem: the theoretical foundation is not mechanized, and there is no  a formal proof
\end{itemize}

\end{frame}

\begin{frame}\frametitle{Our Motivation}

\begin{itemize}
\item automatically searching  auxiliary invariants
\item Formally proving all the things: both the theoretical foundation and case studies
\item A formal proof script as a formal verification product
\end{itemize}
\end{frame}


\begin{frame}\frametitle{An Overview of Our Approach}


\begin{figure}[!t]
\centering %
\vspace{-0.8cm}
\includegraphics[width=1.0\textwidth]{paraVerifierWorkFlow.pdf}
\vspace{-0.6cm}
\caption{The workflow of {\sf paraVerifier}.}
\label{fig:arch}
\end{figure}
\end{frame}

\begin{frame}\frametitle{Some Explanations}

\begin{itemize}
\item {\sf paraVerifier}={\sf invFinder} + {\sf proofGen}

\item protoocl.fl: a small reference instance of the   protocol


\item {\sf invFinder} searches auxiliary invariants automatically

\item   protocol.tbl:  stores the set of ground invariants and a causal relation table

\item {\sf proofGen}: create an Isabelle proof script {\sf protocol.thy} which models and verifies the protocol

\item run Isabelle script to automatically proof-check {\sf protocol.thy}
\end{itemize}

\end{frame}

\begin{frame}\frametitle{Theoretical Foundation-Protocol}
\noindent
A  protocol is formalized as a pair $({\it ini}, {\it rules})$, where
\begin{itemize}
\item ${\it ini}$ is an initialization formula; and
\item  ${\it rules}$ is a set of transition rules. Each rule $r\in {\it rules}$ is defined as
  $g \vartriangleright  S$, where $g$ is a predicate, and $S$ is a
  parallel assignment to distinct  variables $v_i$ with expressions
  $e_i$, where $S$ is  a parallel assignment $S=\{x_i:=e_i | i>0\}$.
\end{itemize}
We write $\mathsf{pre}~r=g$, and $\mathsf{act}~r=S$
  if $r=g \vartriangleright S$.
\end{frame}

\begin{frame}\frametitle{Theoretical Foundation-Causal Relation}



\begin{definition}
We define the following relations
\begin{enumerate}
\item $\mathsf{invHoldForRule_1}~s ~f ~r \equiv  s \models \mathsf{pre}~ r \longrightarrow s \models \mathsf{preCond}~ f ~(\mathsf{act}~ r)$, where $\mathsf{preCond}~S~f=f[x_i:=e_i]$, which substitutes each
occurrence of $x_i$ by $e_i$;
\item $\mathsf{invHoldForRule_2}~s~ f~ r \equiv  s \models f \longleftrightarrow  s \models\mathsf{preCond}~ f~(\mathsf{act}~ r)$;
\item $\mathsf{invHoldForRule_3}~s~ f~ r ~F \equiv$  $\exists f' \in F$ s.t.
$s \models (f' \wedge (\mathsf{pre}~ r)) \longrightarrow s \models \mathsf{preCond} ~f ~(\mathsf{act}   ~r)$;
\item $\mathsf{invHoldForRule}~s~ f~ r ~F$ represents a disjunction of $\mathsf{invHoldForRule_1}$, $\mathsf{invHoldForRule_2}$
and $\mathsf{invHoldForRule_3}$.
%\item $\mathsf{invHoldForRule}~ f~ r ~F \equiv (\mathsf{invHoldForRule_1} ~f
%  ~r) \lor (\mathsf{invHoldForRule_2} ~f ~r) \lor (\mathsf{invHoldForRule_3}~ f~ r~F)$.
\end{enumerate}
\end{definition}
\end{frame}

\begin{frame}\frametitle{Theoretical Foundation - Consistency Relation)}


\begin{definition}
A consistency relation, i.e., $\mathsf{consistent}~ {\it invs} ~{\it ini}~ {\it rules}$,
that holds between a protocol $({\it ini}, {\it rules})$ and
a set of invariants ${\it invs}=\{inv_1,\ldots, inv_n\}$,  is defined as:
%
\begin{itemize}
\item For any invariant ${\it inv} \in {\it invs}$ and state $s$,
if ${\it ini}$ is
evaluated as true at state $s$
(i.e., $\mathsf{formEval}~{\it ini}~s={\it true}$), then ${\it inv}$ is also evaluated as true at the state $s$.

\item For any ${\it inv} \in {\it invs}$,  and $r\in {\it rules}$, and any state $s$,
$\mathsf{invHoldForRule}~{\it inv}~r~{\it invs}$.
\end{itemize}
\end{definition}


\end{frame}




\begin{frame}\frametitle{Theoretical Foundation - Consistent Lemma}

\begin{lemma}\label{consistentLemma}%[(Consistency lemma)]
For a protocol $( ini,  rules)$,
we use $\mathsf{reachableSet}~   ini~ rules$
to denote the set of reachable states of the protocol.
Given a set of invariants $ invs$,
we have
$[| \mathsf{consistent}~  invs ~ ini~  rules;
  s \in \mathsf{reachableSet}~  ini~rules|]\Longrightarrow
  \forall  inv \in invs. \mathsf{formEval}~ inv ~s$
\end{lemma}
\end{frame}



\begin{frame}\frametitle{Key Algorithm of {\sf invFinder}}

\begin{algorithm}\label{alg:invFinder}

\caption{Algorithm: $invFinder$}\label{alg:invfinder}

\KwIn{  Initially given invariants $F$, a protocol $\mathcal{P}=<I,R>$ }

\KwOut{A set of tuples which represent causal relations between concrete rules and invariants: }

{
    $A\leftarrow F$;

    $tuples \leftarrow []$;

    $newInvs \leftarrow F$;

    \While{$newInvs$ is not empty}
    {
   $ f \leftarrow newInvs.dequeue$;

   \For {$r \in R$}
   { $paras \leftarrow \mathsf{Policy}(r,f)$;

      \For {$para \in paras$}
     {$cr \leftarrow \mathsf{apply}(r,para)$;

       $newInvOpt,rel \leftarrow \mathsf{coreFinder}(cr,  f, A)$;

        $tuples \leftarrow tuples @[<r, para, f, rel>]$;

       \If{$newInvOpt \neq NONE$}
        {$newInv \leftarrow \mathsf{get}(newInvOpt)$\;
         newInvs.enqueue(newInv)\;
        $A \leftarrow A \cup \{newInv\}$\;
        }

     }
   }
  }
\Return $tuples$\;
}

%}

\end{algorithm}
 \end{frame}


\begin{frame}\frametitle{More on {\sf invFinder}}


\begin{itemize}
\item trying to construct a consistency relation that
guides the tool {\sf invFinder} to find auxiliary invariants

\item using an oracles that checks whether a ground
formula is an invariant in the small reference model


\item Searching not only auxiliary invariants but also causal relations

\item   protocol.tbl:  storing the searching result
\end{itemize}
 \end{frame}

\begin{frame}\frametitle{A fragment of protocol.tbl}
 \begin{table}[!t]
\centering \caption{A fragment of output of {\sf invFinder}}\label{label-ground-causal relation} % {\tt
%simpMutual.tbl}
\begin{tabular}{|c|c|c|c|c|  }
\hline
  rule& ruleParas&inv&causal relation &   f'  \\
\hline
  .. & ..&.. &..&.. \\

\hline
  crit  & [1]&mutualInv 1 2& invHoldForRule3 &invOnX$_1$~2 \\
\hline
  crit &[2]& mutualInv 1 2& invHoldForRule3 &invOnX$_1$~1  \\
\hline
  crit & [3]& mutualInv 1 2 & invHoldForRule2  & \\
\hline
  .. & ..&.. &..&.. \\

\hline
  crit  & [1]&invOnX$_1$ 1 & invHoldForRule1 &\_ \\
\hline
  crit &[2]& invOnX$_1$ 1 & invHoldForRule1 &\_  \\
\hline
\end{tabular}
\end{table}
\begin{itemize}
\item invariants and causal relations are in concrete form
\item we need parameterized form (or symbolic form)



\end{itemize}
\end{frame}

\end{document}
